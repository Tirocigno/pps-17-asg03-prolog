%\documentclass[10pt,english]{article}
\documentclass[10pt,italian]{article}
\usepackage[utf8]{inputenc} % opzione per caratteri ISO-8859-1, CONSENTE L'USO DELLE ACCENTATE
\usepackage{hyperref}

% MARGINI LARGHI
\textwidth 6.3 in % Width of text line.
    \textheight 9.2 in
    \oddsidemargin 0 in      %   Left margin on odd-numbered pages.
    \evensidemargin 0 in      %   Left margin on even-numbered pages.
    \topmargin 0.2 in
    \headheight 0 in       %   Width of marginal notes.
    \headsep 0 in
    \topskip 0 in
    
\title{\vspace{-70pt}Assignment 03 -- ``Prolog''}
\author{Federico Naldini, matr: 0000852918, email: {\url{federico.naldini3@studio.unibo.it}}\\ repo: {\url{https://github.com/Tirocigno/pps-17-asg03-prolog}}
\date{05/12/2018}}


\begin{document}

\maketitle
\vspace{-30pt}

\section{Description of the code}
L'elaborato che ho scelto di realizzare consiste in una semplice implementazione di \textit{Game of Life}, automa cellulare celebre in letteratura, le cui regole sono descritte alla \href{https://en.wikipedia.org/wiki/Conway\%27s_Game_of_Life#Rules}{seguente pagina}.
Per realizzare il progetto, ho scelto di adottare un pattern \textit{MVC} dove per la parte di \textit{view} e \textit{controller} ho utilizzato i linguaggi Scala e Java, mentre per quanto riguarda il \texttt{model} ho cercato di gestire tutta la complessità del gioco utilizzando la programmazione logica.\\
Il focus dell'assignment è stato nel realizzare la soluzione a un problema ben noto, che ho avuto modo di trattare utilizzando diversi paradigmi di programmazione, applicando un paradigma totalmente differente dai precedenti, cercando di sfruttarne al meglio le potenzialità e di inquadrarne i difetti rispetto alle altre soluzioni.


\section{Techniques used}
\begin{itemize}
	\item \textbf{Minimalità della soluzione proposta:} In linea natura coincisa di Prolog, ho cercato di limitare al massimo le dimensioni del codice: grazie alle possibilità offerte dalla programmazione logica, ho limitato gli elementi fondamentali di cui occorre mantenere lo stato a:
	\begin{itemize}
		\item Le dimensioni della scacchiera di gioco.
		\item Una lista contenente la posizione delle celle in vita durante generazione corrente
		\item Il numero di generazioni trascorse
	\end{itemize}
	\item \textbf{Predicati Tail-Recursion:} Data la necessità di alcune regole di iterare ricorsivamente come le strutture di ciclo presenti nei principali linguaggi di programmazione, ho scelto di realizzare tali regole e i predicati che le compongono come \textit{tail-recursive}, in modo da massimizzare l'efficienza di tale computazione. \texttt{generate\_board} e \texttt{compute\_board} sono due regole realizzate seguendo quanto detto sopra.
	\item \textbf{Integrazione Java-Prolog:} Avendo scelto di mettere a disposizione un'interfaccia grafica scritta in Java al programma, è stato necessario integrare in qualche modo la parte di codice scritta in Prolog con le librerie Java. Per realizzare questa integrazione, ho deciso di sfruttare le API messe a disposizione dalla libreria \texttt{alice.tuprolog}, riuscendo così a ottenere tutta la parte di computazione logica accessibile da codice Java; unica problematica non banale nella sua risoluzione è stata la conversione delle liste Prolog in Java, risolta tramite una combinazione di primitive messe a disposizione dalla libreria per scorrere gli elementi delle liste e codice appositamente scritto per effettuare la conversione del singolo elemento.
	\item \textbf{Performance:} dovendo procedere a una computazione non banale per un elevato numero di celle ad ogni generazione, mi sono scontrato più volte col problema delle performance, trovandomi a dover pensare soluzioni che tenessero conto della complessità computazionale dei predicati presenti nei corpi delle varie regole. Un esempio è sicuramente il predicato \texttt{append\_to\_head}, che inserisce un elemento in testa a una lista, utilizzato al posto delle classiche regole di inserimento che prevedono l'aggiunta del nuovo elemento in coda: non essendo rilevante l'ordine all'interno della lista delle celle in vita, si riduce notevolmente la complessità computazionale dell'inserimento da \textit{O(n)} a un tempo costante.\\
\end{itemize}

\section{Self-evaluation}

 
\end{document}
    
    
